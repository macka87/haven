%%%%%%%%%%%%%%%%%%%%%%%%%%%%%%%%%%%%%%%%%%%%%%%%%%%%%%%%%%%%%%%%%%%%%%%%%%%%%%%%
\documentclass[draft]{llncs}
%%%%%%%%%%%%%%%%%%%%%%%%%%%%%%%%%%%%%%%%%%%%%%%%%%%%%%%%%%%%%%%%%%%%%%%%%%%%%%%% 

\usepackage[usenames]{color}
\usepackage{pslatex}
\usepackage{epsfig}
\usepackage{paralist}
\usepackage{wrapfig}
\usepackage{hyperref}

\overfullrule=5pt

\newcommand{\notice}[1]{\textcolor{red}{#1}}

% \pagestyle{plain}
\pagestyle{empty}

%\pdfpagesattr{/CropBox [92 62 523 728]} % LNCS page made big 

%%%%%%%%%%%%%%%%%%%%%%%%%%%%%%%%%%%%%%%%%%%%%%%%%%%%%%%%%%%%%%%%%%%%%%%%%%%%%%%%

\title{$<$name$>$: An Open Framework for FPGA-Accelerated Functional Verification 
of Hardware
\thanks{This work was supported by the Czech Ministry of Education (project
MSM 0021630528), and the BUT FIT project FIT-11-S-1.}\vspace*{-0mm}}


\author{Michal Kajan%\inst{1}
\and  Ond\v{r}ej Leng\'{a}l%\inst{1}
\and  Marcela \v{S}imkov\'{a}%\inst{1}
}

\institute{ {FIT, Brno University of Technology, Czech Republic}
%\and {Institution 2}
%\and {Institution 3}
}

%%%%%%%%%%%%%%%%%%%%%%%%%%%%%%%%%%%%%%%%%%%%%%%%%%%%%%%%%%%%%%%%%%%%%%%%%%%%%%%%
\begin{document} 
%%%%%%%%%%%%%%%%%%%%%%%%%%%%%%%%%%%%%%%%%%%%%%%%%%%%%%%%%%%%%%%%%%%%%%%%%%%%%%%%

\maketitle

\vspace*{-0mm}\begin{abstract}Functional verification is a widespread technique 
for checking whether a hardware system satisfies a given correctness specification. 
As the complexity of modern computer systems rises rapidly, it is a challenging 
task to find appropriate acceleration techniques for this process.
In this paper we present $<$name$>$, a freely available open functional
verification framework that exploits the field-programmable gate array (FPGA)
technology for cycle-accurate acceleration of simulation runs.
$<$name$>$ moves the design under test (DUT) together with transaction-based
interface drivers and monitors from the simulator into an FPGA environment built
upon the NetCOPE platform.
Our experiments show that the achieved acceleration is proportional to the
complexity of the verified system, with the peak value being over 130 times.
\end{abstract}

%===============================================================================
\vspace*{-0mm}\section{Introduction}\vspace*{-0mm}
%===============================================================================
Today's highly competitive market of consumer electronics is very sensitive to the 
time it takes to introduce a new product (the so-called time to market). This has 
driven the demand for fast, efficient and cost-effective methods of verification 
of hardware systems. Simulation-based approaches like functional verification 
suffer from the fact that software simulation of inherently parallel hardware is 
extremely slow when compared to the speed of real hardware. The gap between the 
speed of simulation and the speed of real hardware widens with the increasing 
complexity of the hardware design. An effort to increase efficiency and speed of 
simulation or functional verification poses a considerable challenge not only for 
research teams but also for commercial sphere (scope, area, field) (e.g. Mentor 
Graphics' Veloce technology~\cite{mentor-veloce}). As representatives of the first mentioned community 
we introduce an open framework that exploits the inherent parallelism of hardware 
designs to accelerate functional verification of these designs by targeting 
special components of the verification environment to the FPGA. This is possible 
because the generic nature of verification methodologies (OVM, UVM) and 
transaction-based communication among their subcomponents make it possible to 
transparently move these subcomponents to a specialized hardware, while 
maintaining the same level of readability to verification engineers.

\notice{Here is to be related work (as a part of introduction, not separate section)}
During the development of the framework, we noticed the work of Huang
\textit{et al}~\cite{huang-soc-hw-verification}, but because \notice{sth}, we
believe that our contribution is still valuable.


%===============================================================================

%===============================================================================
\vspace*{-0mm}\section{Design of a Verification Framework}\vspace*{-0mm}
%===============================================================================

%===============================================================================

%===============================================================================
\vspace*{-0mm}\section{Experimental Results}\vspace*{-0mm}
%===============================================================================

%===============================================================================

%===============================================================================
\vspace*{-0mm}\subsection{Bug hunting}\vspace*{-0mm}
%===============================================================================

%===============================================================================

%===============================================================================
\vspace*{-0mm}\section{Conclusion}\vspace*{-0mm} \label{sec:conclusion}

Here should be the conclusion.

\notice{Here is to be future work (as a part of conclusion, not separate section)}

%===============================================================================

%%%%%%%%%%%%%%%%%%%%%%%%%%%%%%%%%%%%%%%%%%%%%%%%%%%%%%%%%%%%%%%%%%%%%%%%%%%%%%%%

% \bibliographystyle{splncs03}
% \bibliographystyle{plain}

%\bibliography{literature}

{
\begin{thebibliography}{10}

%\fontsize{9pt}{10pt}\selectfont

%\vspace*{-2mm}

\bibitem{mentor-veloce}
Mentor Graphics.
\newblock Veloce.\\
\newblock \url{http://www.mentor.com/products/fv/emulation-systems/veloce/}

\bibitem{huang-soc-hw-verification}
C.-Y.~Huang, Y.-F.~Yin, C.-J.~Hsu, T.~B.~Huang, and T.~M.~Chang.
\newblock SoC HW/SW Verification and Validation.
\newblock In {\em Proc. of ASPDAC'11}, IEEE, 2011.

\bibitem{parosh-rtmc}
P.~A. Abdulla, B.~Jonsson, P.~Mahata, and J.~d'Orso.
\newblock Regular Tree Model Checking.
\newblock In {\em Proc. of CAV'02}, LNCS 2404, Springer, 2002.

\bibitem{antichain}
A.~Bouajjani, P.~Habermehl, L.~Hol\'{i}k, T.~Touili, and T.~Vojnar.
\newblock {Antichain-based Universality and Inclusion Testing over
  Nondeterministic Finite Tree Automata}.
\newblock In {\em Proc. of CIAA'08}, LNCS~5148, Springer, 2008.

\bibitem{bouajjani-artmc}
A.~Bouajjani, P.~Habermehl, A.~Rogalewicz, and T.~Vojnar.
\newblock Abstract Regular Tree Model Checking.
\newblock {\em ENTCS}, 149, Elsevier, 2006.

\end{thebibliography}
}

%%%%%%%%%%%%%%%%%%%%%%%%%%%%%%%%%%%%%%%%%%%%%%%%%%%%%%%%%%%%%%%%%%%%%%%%%%%%%%%%
\end{document}
%%%%%%%%%%%%%%%%%%%%%%%%%%%%%%%%%%%%%%%%%%%%%%%%%%%%%%%%%%%%%%%%%%%%%%%%%%%%%%%%
